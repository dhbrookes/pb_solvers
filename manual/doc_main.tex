% !TEX encoding = UTF-8 Unicode
\documentclass[11 pt, a4paper,titlepage]{report}
%\documentclass[11pt, oneside]{memoir}   
\usepackage[margin=0.9in]{geometry}             

\RequirePackage{fix-cm}
\RequirePackage{titlesec}
\RequirePackage{titletoc}  

%% --------------- Load packages for large documents. ------------------
%% Hyperref is used to setup the hyperlinking between 
%% chapter/section/subsection and contents, citing and references,
%% and footnotes. \widowpenalty and \clubpenalty are used to suppress
%% WINDOWS (the last line of a paragraph typed as the first line of a 
%% new page) and ORPHANS (the first line of a paragraph as the last
%% line of a page).
\usepackage{chapterbib}
% For Bib by Chapter, bibtex the .aux file for each chapter after latex of main
%\usepackage{cite}%                %conflicting with hyperref
\usepackage[colorlinks,%
linkcolor=black,%
anchorcolor=black,%
citecolor=black]{hyperref}%         %setup hyderlinking
\widowpenalty=10000%
\clubpenalty=10000%                % suppress widows and orphans



%\renewcommand{\chaptername}{Chapter}
\titleformat{\chapter}[display]%
  {\bfseries\fontsize{16pt}{19.2pt}\selectfont}
  {\chaptertitlename\ \thechapter}{\baselineskip}{}
\titleformat{\section}[hang]%
  {\bfseries\fontsize{14pt}{16.8pt}\selectfont}
  {\thesection}{\baselineskip}{}
\titleformat{\subsection}[hang]%
  {\bfseries\fontsize{12pt}{14.4pt}\selectfont}
  {\thesubsection}{\baselineskip}{}
  
  
\usepackage{array}
\usepackage{amssymb}
\usepackage{amsmath}
\usepackage{caption}

\usepackage{rotating}
\usepackage{chapterbib}
\usepackage{graphicx}	
\usepackage{ctable}
\usepackage{multirow}
\usepackage[normalsize]{subfigure} 										
	

%% This is for the code segments of manual!!
\usepackage{color}
\usepackage{listings}
\lstdefinestyle{ MyBash}{language=bash,                % choose the language of the code
basicstyle=\footnotesize,       % the size of the fonts that are used for the code
numbers=left,                   % where to put the line-numbers
numberstyle=\footnotesize,      % the size of the fonts that are used for the line-numbers
stepnumber=1,                   % the step between two line-numbers. If it is 1 each line will be numbered
numbersep=5pt,                  % how far the line-numbers are from the code
backgroundcolor=\color{white},  % choose the background color. You must add \usepackage{color}
showspaces=false,               % show spaces adding particular underscores
showstringspaces=false,
commentstyle=\color{red},
keywordstyle=\color{blue},
showtabs=false,                 % show tabs within strings adding particular underscores
frame=single,           % adds a frame around the code
tabsize=2,          % sets default tabsize to 2 spaces
captionpos=b,           % sets the caption-position to bottom
breaklines=true,        % sets automatic line breaking
breakatwhitespace=false,    % sets if automatic breaks should only happen at whitespace
escapeinside={\%*}{*)}          % if you want to add a comment within your code
}

\lstdefinestyle{MyCpp}{ language=C++,                % choose the language of the code
basicstyle=\footnotesize,       % the size of the fonts that are used for the code
numbers=left,                   % where to put the line-numbers
numberstyle=\footnotesize,      % the size of the fonts that are used for the line-numbers
stepnumber=1,                   % the step between two line-numbers. If it is 1 each line will be numbered
numbersep=5pt,                  % how far the line-numbers are from the code
backgroundcolor=\color{white},  % choose the background color. You must add \usepackage{color}
showspaces=false,               % show spaces adding particular underscores
showstringspaces=false,
basicstyle=\ttfamily,
keywordstyle=\color{blue}\ttfamily,
stringstyle=\color{red}\ttfamily,
commentstyle=\color{green}\ttfamily,
morecomment=[l][\color{magenta}]{\#},
showtabs=false,                 % show tabs within strings adding particular underscores
frame=single,           % adds a frame around the code
tabsize=2,          % sets default tabsize to 2 spaces
captionpos=b,           % sets the caption-position to bottom
breaklines=true,        % sets automatic line breaking
breakatwhitespace=false,    % sets if automatic breaks should only happen at whitespace
escapeinside={\%*}{*)}          % if you want to add a comment within your code
}


\usepackage{fancyhdr}

%\setlength{\droptitle}{-5em}   % This is your set screw
\usepackage[parfill]{parskip}
\setlength{\parskip}{1pt}
\setlength{\parsep}{1pt}
\setlength{\topsep}{0pt}
\setlength{\partopsep}{0pt}

\graphicspath{ {./figures/} }

\pagestyle{fancy}
\fancyhf{}
\fancyhead[R]{ PB-[S]AM manual}
\fancyfoot[R]{\thepage\ }



\renewcommand\contentsname{Table of Contents}



\newcommand{\param}[1]{$\textless\texttt{#1}\textgreater$}
\newcommand\T{\rule{0pt}{3.0ex}}       % Top strut
\newcommand\B{\rule[-2ex]{0pt}{0pt}}


%%%%%%%%%%%%%%%%%%%%%%%%%%%%%%%%%%%%%%%%%%%%%%%%%%%%%%%%%%%%%%%%%%
%%%%%%%%%%%%%%%%%%%%%%%%%%%%%%%%%%%%%%%%%%%%%%%%%%%%%%%%%%%%%%%%%%


\begin{document}



\begin{titlepage}
\pagenumbering{gobble}
\begin{center}
\vspace{5pt}

%\title
	{\huge \textbf{Reference Manual}}\\ 
	\vspace{0.2cm}
	{\huge Poisson Boltzmann Analytical Model (PB-AM)}\\ 
\vspace{5pt}

\includegraphics[scale=0.8]{cover}

%\author
	Enghui-Yap\\  
	Albert Einstein College of Medicine\\ 
\vspace{5pt}	
	
	Lisa Felberg\\ 
	Marielle Soniat\\
	David Brookes \\ 
	Teresa Head-Gordon\\ 
	University of California, Berkeley  
\vspace{5pt}

For more information, please visit http://thglab.berkeley.edu \\
\vspace{5pt}
\end{center}
\end{titlepage}



\textbf{Cover Illustration:} An exploded view of a Brome Mosaic Virus capsid 
composed of T = 1 particles (PDB: 1YC6), 
represented as collections of overlapping spheres, is shown. 
PB-SAM is a new semi-analytical approach to efficiently solve the 
linearized Poisson?Boltzmann equation using multipole formalisms for overlapping spheres. 
The background shows the potential profile for an array of 601YC6 monomers computed using this method. \\



\textbf{Recommended Citations:} \\

When citing PB-AM in the literature, the following citation should be used

I. Lotan and T. Head-Gordon (2006). An analytical electrostatic model for salt screened interactions between multiple proteins  J. Chem. Theory Comput 2, 541-555. \\ 


When citing PB-SAM in the literature, the following citations should be used

1.  E.-H. Yap and T. Head-Gordon (2010). New and efficient Poisson-Boltzmann solver for interaction of multiple proteins  J. Chem. Theory Comput. (Journal cover) 6, 2214-2224.

2.  E.-H. Yap and T. Head-Gordon (2013). Calculating the bimolecular rate of protein?protein association with interacting crowders.  J. Chem. Theory Comput. 9(5), 2481-2489.

3.   O. N. Demerdash, E.-H. Yap and T. Head-Gordon (2014). Advanced potential energy surfaces for condensed phase simulation.  Ann. Rev. Phys. Chem. 65, 149-174.   \\



\textbf{Acknowledgments:} Research support from NIH and DOE is gratefully acknowledged. \\


\tableofcontents

\pagenumbering{arabic}
\setcounter{page}{1}


\chapter{Introduction}



The Poisson-Boltzmann Analytical Model solves the linearized Poisson-Boltzmann Equation (PBE) for systems hitherto not possible using traditional PBE solvers. This manual describes the method and its associated suite of programs. The PBE software suite is licensed as a collection of freely available program under a GPL license. 


\section{PB-AM} The first general analytical solution for computing the screened electrostatic interaction between large numbers of macromolecules of arbitrarily complex charge distributions, assuming they are well described by spherical low dielectric cavities in a higher dielectric medium in the presence of a Debye-H{\"u}ckel treatment of salt. The method exploits multipole expansion theory for the screened Coulomb potential such that it can describe direct charge-charge interactions and all higher-order cavity polarization effects between low dielectric spherical cavities containing their charges, while treating these higher order terms correctly at all separation distances. The analytical solution is general to arbitrary numbers of macromolecules, is efficient to compute, provides for the first time the ability to provide new benchmarks for other numerical solutions to the linearized Poisson-Boltzmann equation. A number of utilities are described below that use PB-AM results.

\section{Physical Calculations} The PB-AM results allow for fast calculation of physically important quantities. Specifically this program can calculate the interaction energy of, and the force and torque applied to each molecule. These results may be written into a file or printed to the terminal.

\section{Brownian Dynamics} This package also implements dynamics simulations using the Brownian protocol developed by Ermak and McCammon\footnote{Ermak, D. L.; McCammon, J. A. \textit{J. Chem. Phys.} 1978, \textit{69}, 1352\---1360.}. At each time step, the translation and rotation due to the applied force and torque, respectively, are calculated and added to random components of motion. The user can specify spatial and temporal termination conditions for the simulation and write the trajectory to specified files.

\section{Electrostatics} Another possible output of PB-AM, the user can specify a configuration of an arbitrary number of molecules and get a 2-dimensional or 3 dimensional potential landscape of the system. We have provided example plotting tools, and the 3-dimensional output may be uploaded in VMD for visualization.



\clearpage

\chapter{Theory}



\section{The Model}



\subsection{PB-AM formulation}

PB-AM is an analytical solution to the linearized Poisson-Boltzmann equations for multiple spherical objects of arbitrary charge distribution in an ionic solution.  The linearized Poisson-Boltzmann equation is given as:

\[ \nabla [\epsilon(r) \nabla\phi(r)] - \epsilon(r) \kappa^2\phi(r) = 4 \pi \rho(r) \] %∇[ϵ(r)∇Φ(r)]-ϵ(r) κ^2 Φ(r)=4πρ(r)

\[ \phi_{out}^{(i)}= \phi_{out}^{(i)} \biggr |_{r=a_i } \]

\[\epsilon_s \frac{\partial \phi_{out}^{(i)}}{\partial r} =   \epsilon_s \frac{\partial \phi_{out}^{(i)}}{\partial r} \biggr |_{r=a_i } \]

Exploiting fast-multipole methods, this boundary value problem can be reduced to the following system of linear equations.  

\[ A = \Gamma \cdot (\Delta \cdot T \cdot A + E) \]

A(i) represents the effective multipole expansion of the charge distributions of molecule (i). E(i) is the free charge distribution of molecule (i). $\Gamma$ is a dielectric boundary-crossing operator, $\Delta$ a cavity polarization operator, T an operator that transforms the multipole expansion to a local coordinate frame.  More details on the method are available in Lotan, Head-Gordon (2006). \\

\subsection{Physical Calculations}

From the above formulation, computation of the interaction energies ($\Omega^{(i)}$) is given as follows:

\[\Omega^{(i)}=\frac{1}{\epsilon_s}  \langle  T \cdot A^{(i) } ,  A^{(i) } \rangle \]

Where $\langle  M, N \rangle$ denotes the inner product. When energy is computed, forces follow as:

\[ \textbf{F}^{(i)} = \nabla_i \Omega^{(i)}=\frac{1}{\epsilon_s} [ \langle \nabla_i \,T \cdot A^{(i) } ,  A^{(i) } \rangle +  \langle T \cdot A^{(i) } ,   \nabla_i \, A^{(i) } \rangle ]\]


% Additonally, the torque $\tau^{(i)}_j$ on molecule $i$ due to a charge $q^{(i)}_j$ is calculated as follows:
The method to calculate the torque $\boldsymbol{\tau}^{(i)}$ on molecule is outside the scope of this manual, but is discussed extensively in Lotan, Head-Gordon (2006).

\subsection{Brownian Dynamics}

Brownian dynamics simulations are implemented by treating each molecule as a Brownian particle experiencing a conservative force $\textbf{F}^{(i)}$ and torque $\boldsymbol{\tau}^{(i)}$, as well as friction and random force due to the solvent. The translation $\Delta r_i$ and rotation $\Delta \theta_i$ for a time step $\Delta t$ are then given by

\[\Delta r^{(i)} = \frac{D_{i, trans} \Delta t}{k_B T} \textbf{F}^{(i)} + \textbf{S}_i(\Delta t)\]
\[\Delta \theta^{(i)} = \frac{D_{i, rot} \Delta t}{k_B T} \boldsymbol{\tau}^{(i)} + \boldsymbol{\Theta}_i(\Delta t)\]

where $D_{i, trans}$ and $D_{i, rot}$ are the translation and rotational diffusion coefficients for molecule $i$, respectively and $\textbf{S}_i(\Delta t)$ and $\boldsymbol{\Theta}_i(\Delta t)$ are the stochastic components of translation and rotation, respectively, which have the following properties:

\[\langle \textbf{S}_i \rangle=0, \qquad \langle \textbf{S}_i^2 \rangle=2D_{i, trans}\Delta t\]
\[\langle \boldsymbol{\Theta}_i \rangle=0, \qquad \langle \boldsymbol{\Theta}_i^2 \rangle=2D_{i, rot}\Delta t\]

\subsection{Electrostatics}

\clearpage


\section{Installation}

\subsection{PB-AM Installation}

TODO

% To install PB-AM from the source code, first pull the latest version from the github/bitbucket site and type the following into the command line: \\
% \begin{lstlisting}[style = MyBash]
% >> make mpe
% \end{lstlisting}
% This should make the executable, mpe, and place it in the bin directory of the source code




%%%%%%%%%%%%%%%%%%%%%%%%%%%%%%%%%%%%%%%%%%%%%%%%%%%%%%%%%%%%%%%%%%
%%%%%%%%%%%%%%%%%%%%%%%%%%%%%%%%%%%%%%%%%%%%%%%%%%%%%%%%%%%%%%%%%%
\subsection{PB-AM: Example input files and input file information}

\subsubsection{Program option}

The program executable requires an input file as a command line parameter. The input file contains the various arguments and parameters that one may wish to set when running the program. Each line of the input file contains a keyword followed by a variable number of whitespace-delimited parameters, e.g.: \\

\texttt{keyword1\qquad param1\qquad param2} \\
\texttt{keyword2\qquad param1\qquad param2\qquad param3} \\

Each keyword is described in the table below, along with its associated parameters.


\newcommand{\param}[1]{$\textless\texttt{#1}\textgreater$}
\newcommand\T{\rule{0pt}{3.5ex}}       % Top strut
\newcommand\B{\rule[-2ex]{0pt}{0pt}}

\newlength{\colthree}
\setlength{\colthree}{10.1cm}
\newlength{\coltwo}
\setlength{\coltwo}{2.9cm}

\begin{tabular}{ c | l | l  }
    \textbf{Keyword} & \textbf{Parameters} & \textbf{Description} \\ \hline
\T runname & \param{name} & \parbox[t]{\colthree}{\param{name} is desired internal name of this run.} \\
\T pqr & \param{fpath} & \parbox[t]{\colthree}{Provide input PQR file at \param{fpath}.} \\
\T xyz & \param{fpath} & \parbox[t]{\colthree}{Provide input XYZ file at \param{fpath}.} \\
\T salt & \param{con} & \parbox[t]{\colthree}{Set salt concentration in the system to \param{con}.}\\
\T temp & \param{T} & \parbox[t]{\colthree}{Set system temperature to \param{T}}\\
\T idiel & \param{ival} & \parbox[t]{\colthree}{Set the interior dielectric constant to \param{ival}.} \\
\T sdiel & \param{sval} & \parbox[t]{\colthree}{Set the interior dielectric constant to \param{sval}.} \\
\T pbc & \param{boxlength} & \parbox[t]{\colthree}{Set size of periodic box to \param{boxlength}.}\\
\T random & \param{seed} & \parbox[t]{\colthree}{Seed the internal random number generator with \param{seed}.} \\
\T attypes & \param{numtypes} & \parbox[t]{\colthree}{Set the number of different atom types to \param{numtypes}.}\\
\T type & \parbox[t]{\coltwo}{\param{idx} \param{ct} \param{movetype} \param{dtr} \param{drot}} & \parbox[t]{\colthree}{Set attributes of an atom type, where \param{idx} is the integer id of this type, which can be 1 to \param{numtypes} (above). \param{ct} is the number of atoms of this type in the system and \param{movetype} describes the way this type is allowed to move in a dynamics run (\texttt{move}, \texttt{rot}, or \texttt{stat}). If \param{movetype} is \texttt{move}, then a translational diffusion coefficient \param{dtr} and a rotational diffusional coefficient \param{drot} are required. If \param{movetype} is \texttt{rot} then just \param{drot} is required. \B}\\
\hline
\T runtype electrostatics & \param{gridpts} & \parbox[t]{\colthree}{Will run electrostatics calculations. \param{gridpts} is an optional integer describing the number of evenly spaced points in each dimension to perform calculations on.}\\
\T dx & \param{fname} & \parbox[t]{\colthree}{For electrostatics. Will write the results of electrostatics calculations for every 3D grid point to \param{fname}.} \\
\T gridct & \param{ct} & \parbox[t]{\colthree}{For electrostatics. \param{ct} is the number of 2D grids to output.} \\
\T grid2d & \parbox[t]{\coltwo}{\param{idx} \param{fname} \param{axis} \param{val} } & \parbox[t]{\colthree}{For electrostatics. Set attributes of a grid output where \param{idx} is the integer id of this grid, which can be 1 to \param{ct} (above). Will write output of calculations for a cross section along \param{axis} (\texttt{x}, \texttt{y}, or \texttt{z}) at \param{value}.\B} \\
\hline
\T runtype dynamics & & \parbox[t]{\colthree}{Will perform a brownian dynamics run.\B}  \\
\T termct & \param{ct} & \parbox[t]{\colthree}{Set number of termination conditions to \param{ct}.}  \\
\T termcombine & \param{andor} & \parbox[t]{\colthree}{Set how termination conditions will be combined. \param{andor} should be \texttt{and} or \texttt{or}.} \\
\T term & \parbox[t]{\coltwo}{\param{idx} \param{type} \param{val}} & \parbox[t]{\colthree}{Set attributes of a termination condition where \param{idx} is the integer id of this condition, which can be 1 to \param{ct} (above). \param{type} can be \texttt{time}, \texttt{x}, \texttt{y}, \texttt{z}, or \texttt{r} and \param{val} is the value where the simulation will terminate. \B} \\
\hline
  \end{tabular}

\subsubsection{System inputs}

From the single mpe executable, multiple types of calculations can be performed. Generally, all the programs require a computation flag, and a PDB or PQR file name.  If a PDB file is chosen the input is read and atoms are assigned partial charges according to the file charges\_OPLS, located in each of the test directories. A PQR file can be generated from the online site or the software can be downloaded:  \\

http://nbcr-222.ucsd.edu/pdb2pqr\_1.9.0/  \\
http://www.poissonboltzmann.org/docs/pdb2pqr-installation/ \\

It may also be formatted manually. The general format of a PQR file is as follows, and is whitespace-delimited: \\

\textbf{recName  serial  atName  resName  chainID  resNum  X  Y  Z  charge rad }\\

  \begin{tabular}{ c | l  }
    \textbf{Delimiter} & \textbf{Description} \\ \hline
recName 	&	A string that should either be ATOM or HETATM. \\
serial 	&	An integer that provides the atom index \\
atName 	&	A string that provides the atom name.\\
resName	&	A string that provides the residue name. \\
chainID	&	An optional string that provides the chain ID of the atom.\\
residueNumber  & An integer that provides the residue index.\\
X Y Z	& Three floats that provide the atomic coordinates.\\
charge	& A float that provides the atomic charge (in electrons). \\
Rad		& A float that provides the atomic radius (in \AA…).\\
    \hline
  \end{tabular}


\clearpage


\include{chapters/pbsam}

\chapter{Analysis Tools}

\section{Viewing PQR in VMD}

%Load \<yourfile\>\.all\.pqr
Load pqr file
\begin{lstlisting}[style = MyBash]
>> set selall [atomselect top "all"]
\end{lstlisting}

To center the pqr
\begin{lstlisting}[style = MyBash]
>> $selall moveby {-x -y -z}
\end{lstlisting}
where x, y, and z are half the box length

Graphical representations:
To view the charges inside the CG center, from the toolbar, select Graphics \textgreater \,
Representations. In the selected atoms, type
\begin{lstlisting}[style = MyBash]
not name X
\end{lstlisting}

Change the coloring method to Charge, and the Drawing Method to VDW. Then select the 
Create Rep button, and in the selected atoms, type 
\begin{lstlisting}[style = MyBash]
not name X
\end{lstlisting}

 Change the Drawing Method to VDW and the Material to Transparent. 
 The Graphical Representations and final images are in figure below.

\begin{figure}[!htbp]
  \centering
  \begin{minipage}[b]{0.3\textwidth}
    \includegraphics[width=\textwidth]{vmd_graph_rep_pqr}
    \caption{Graphics}
  \end{minipage}
  \hfill
  \begin{minipage}[b]{0.65\textwidth}
    \includegraphics[width=\textwidth]{vmd_4sp}
    \caption{Final view}
  \end{minipage}
\end{figure}

\section{Viewing Electrostatics in VMD}

To load electrostatic results: File \textgreater \, New Molecule. In the window that appears, toggle the \textit{Load files for:} 
to select the currently loaded PQR file. Then select \textit{Browse} to find the location of the dx file. Once found, hit 
the \textit{Load} button and let the dx file load.

\begin{figure}[!htbp]
  \centering
  \begin{minipage}[b]{0.3\textwidth}
    \includegraphics[width=\textwidth]{vmd_graph_rep_dx}
    \caption{DX Graphics}
  \end{minipage}
  \hfill
  \begin{minipage}[b]{0.65\textwidth}
    \includegraphics[width=\textwidth]{vmd_4sp_dx}
    \caption{DX representation}
  \end{minipage}
\end{figure}

 \clearpage

%\section{Mean First Passage Time (MFPT)}
%
%collect FPT from complete.dat
%\begin{lstlisting}[style = MyBash]
%	>> cat <yourpath>/complete.dat | awk '{print $7}' >  fpt.txt
%\end{lstlisting}
%run lognorm.py 
%\begin{lstlisting}[style = MyBash]
%	>> python lognorm.py  <yourpath>/fpt.txt
%\end{lstlisting}
%The final line of the output gives average and standard deviation for first passage times
%based on the lognormal probability distribution.
%The other lines output shape parameters describing the skew and kurtosis of the PDF
%It can also plot the histogram of FPT.
%
%
%To Do : get diffusion constant from fpt.txt as well


\section{2D ESP plots}
From the electrostatic option in PB-AM, a file is created with the following format: \\

\begin{lstlisting}[style = MyBash]
# Data from PBAM Electrostat run
# My runname is barnase.x.0.dat
units kT
grid 200 200 
axis x 0 
origin -51.2204 -51.2204
delta 0.512204 0.512204
maxmin 1.35396 0
   0.2352107     0.2360552     0.2368904     0.2377159
\end{lstlisting}

\medskip

In our \texttt{tools} directory we have provided a python script for plotting this potential.
Change filename within .py file, and the desired output name.
\begin{lstlisting}[style = MyBash]
>> python potential_plot
\end{lstlisting}

\medskip

creates a JPG file of the cross sectioned potential, like the one given below:

\begin{figure}[!htbp]
  \centering
    \includegraphics[scale=0.3]{pot_x_0}
    \caption{Potential plot}
\end{figure}

\section{3D ESP plots}
From the electrostatic option in PB-AM, a file is created with the following format: \\

\begin{lstlisting}[style = MyBash]
# Data from PBAM Electrostat run
# My runname is electro_map.out and units kT
grid 10 10 10
origin -4 -9 -9
delta 0.8 1.8 1.8
  0.00000   0.00000  -2.90000 -5.899581 
\end{lstlisting}

\medskip

In our \texttt{tools} directory we have provided a python script for plotting this potential.
Change filename within .py file, and the desired output name.
\begin{lstlisting}[style = MyBash]
>> python plot_3d_surf
\end{lstlisting}

\medskip

creates a JPG file of the cross sectioned potential, like the one given below:

\begin{figure}[!htbp]
  \centering
    \includegraphics[scale=0.4]{4sp_surf}
    \caption{Molecule surface potential plot}
\end{figure}

	
%\section{Comparison of Displacement Due to Electrostatics versus Diffusion}	
%
%Because Brownian dynamics directly calculated displacements, 
%it is easier to compare relative displacements rather than relative forces.
%
%Separate the displacements due to Electrostatics and Diffusion into their own files.
%In the same directory as bdtest.out :
%\begin{lstlisting}[style = MyBash]
%	>> bash cfDisplacements.sh
%\end{lstlisting}
%Currently set to use only the final completed trajectory in bdtest.out
%This creates two files : dispES.out and dispRand.out
%In the same directory:
%\begin{lstlisting}[style = MyBash]
%	>> python avgDisp.py
%\end{lstlisting}
%This outputs the average displacement in Angstroms due to electrostatics and diffusion, in that order.
%
%
%
%\section{Length Scaling of FPT}
%
%How does the MFPT change with arbitrary choice of length scale?
%
%\begin{lstlisting}[style = MyBash]
%	>> bash lengthScale.sh
%\end{lstlisting}
%
%This script re-analyzes the trajectory for FPT with distances of 10, 20, ... \AA{}.





\chapter{Common Errors}

\section{Errors while reading input file}
Our file reader catches some errors, like files not exisiting or
being unable to open them. Look at the verbose stream of the
program to make sure all your flags are being caught, and check the
stream for errors as well.

\section{Segmentation Fault while reading input file}
Our file reader is not very robust.
Make sure there is only a single space between keywords and options.
Make sure that all options are specified for a given keyword.

\section{Initial configuration errors}
Many issues may arise if your molecules are overlapping, which is not allowed in the PB solvers' models.
The center of the molecule will be placed at the positions given by each XYZ file.
Check the printed out PQR file for the initial configuration, which you can load into VMD.
Try changing the xyz file(s).
This error may also appear if the box length specified with the pbc keyword is too small.
Try increasing the box length.












\phantomsection
\addcontentsline{toc}{chapter}{Bibliography}
\bibliographystyle{jacs}
\bibliography{pbsam}


\end{document}

